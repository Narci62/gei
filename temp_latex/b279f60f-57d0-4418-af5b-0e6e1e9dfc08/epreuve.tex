\documentclass[12pt,a4paper]{article}
\usepackage[utf8]{inputenc}
\usepackage[french]{babel}
\usepackage[T1]{fontenc}
\usepackage{amsmath,amssymb,amsfonts}
\usepackage{graphicx,multicol}
\usepackage{pgfplots}
\pgfplotsset{compat=1.15}
\usepackage{mathrsfs}
\usetikzlibrary{arrows,patterns}
\usepackage{fourier}
\mathversion{bold}
\everymath{\displaystyle}
\usepackage[left=2cm,right=1.5cm,top=1cm,bottom=1.5cm]{geometry}
\usepackage{fancyhdr}
\pagestyle{fancy}
\renewcommand{\headrulewidth}{0pt}
\renewcommand{\footrulewidth}{0pt}
\lhead{}\chead{}\rhead{}
\lfoot{}\cfoot{}\rfoot{}
\begin{document}
\noindent\begin{tabular}{|p{5cm}|p{7cm}|p{4cm}|}
\hline 
\textbf{IESG:} & \textbf{Devoir Surveillé} &  \textbf{Classe : 6 ème}\\ 
\hline 
\textbf{AN/SC : } &\textbf{Épreuve de MATHS} &  \textbf{ Durée : 2h}\\ 
\hline 
\end{tabular} \\

\noindent\textbf{\underline{EXERCICE 1}}  \textit{(8pts)}\\\indent Ton établissement organise chaque année des compétitions de football en prélude aux journées culturelles. Cette année, pour prendre compte l'équité genre, Le proviseur demande à ton Professeur de sports de constituer des équipes mixtes, c'est-à-dire des équipes composées de garçons et de filles. Toutes les équipes doivent comporter un même nombre de filles et un même nombre de garçons; le nombre de filles peut être différent du nombre de garçons. $80$ filles et $150$ garçons sont aptes pour jouer et tout(e) élève apte doit jouer dans une équipe.\\
Le professeur de sports voudrait former le plus grand nombre possible d'équipes mixtes. Il sollicite ton aide, toi élève en classe de quatrième($4^{eme}$), pour déterminer le nombre d'équipes mixtes qu'il pourra constituer, déterminer aussi la composition de chaque équipe en fille et en garçon et ensuite le nombre totale d'élèves par équipe.
\begin{center}
\begin{tabular}{|p{3cm}|p{3cm}|p{3cm}|p{3cm}|p{3cm}|}
\hline 
\textbf{Critères} & CM 1 & CM 2 & CM 3 & CP \\ 
\hline 
\textbf{Barème} & 2 pts & 2 pts & 2 pts & 2 pts\\ 
\hline 
\end{tabular} 
\end{center}\noindent\textbf{\underline{EXERCICE 2}}  \textit{(6pts)}\\\noindent\textbf{\underline{Partie A}}  \textit{(2pts)}Complète le texte suivant en utilisant les lettres:

Le $PPCM$ de deux nombre entiers naturels est le produit des $\cdots(a)\cdots$ des décompositions des deux nombres, chaque facteurs étant affecté du  plus grand $\cdots(b)\cdots$ apparu dans les deux décompositions. Si un point est équidistant de deux droites sécantes, alors il appartient à la $\cdots(c)\cdots$ de l' $\cdots(d)\cdots$ formés par ces deux droites. Le centre de gravité d'un triangle est situé au $\frac{2}{3}$ de chaque $\cdots(e)\cdots$ à partir du $\cdots(f)\cdots$

\noindent\textbf{\underline{Partie B}}  \textit{(2pts)}Réponds par \textbf{Vrai} ou \textbf{Faux}:
\begin{enumerate}
\item Par une symétrie, un segment à pour image un segment de même longueur.
\item Le polygone régulier est un polygone inscriptible dans un cercle ayant ses cotés de même longueur.
\item Une fraction est irréductible lorsque le $PGCD$ du numérateur et du dénominateur est égal à $1$.
\item Le point de concours des bissectrices des angles d'un triangle est appelé centre du cercle circonscrit à ce triangle.
\end{enumerate}

\noindent\textbf{\underline{Partie C}}  \textit{(2pts)}Choisis la bonne réponse:
\item  Le $PPCM (12;48)$ est égal à :
\setlength{\multicolsep}{0pt}
\begin{multicols}{3}
\begin{enumerate}
\item $24$ 
\item $48$ 
\item $12$ 
\end{enumerate}
\end{multicols}
\item Le point de concours des médianes des côtés d’un triangle est appelé:
\setlength{\multicolsep}{0pt}
\begin{multicols}{3}
\begin{enumerate}
\item Centre du cercle inscrit
\item Centre de gravité
\item Orthocentre.
\end{enumerate}
\end{multicols}
\end{enumerate}

\noindent\textbf{\underline{EXERCICE 3}}  \textit{(6pts)}\\$A/$ $CEG$ est un triangle tel que $CE=3cm$ ; $CG=4cm$ et $EG=5cm$. Le point $A$ est le milieu du segment $[CE]$ et $B$ milieu de $[CG]$. $H$ est le pied de la hauteur issue du sommet $C$.
\begin{enumerate}
\item Fais une figure bien soignée et codée. 
\item Calcule $CE^2$, $CG^2$ et $EG^2$. 
\item Justifie que le triangle $CEG$ est un triangle rectangle en $C$. 
\item Calcule les distances $CH$ et $AB$.
\end{enumerate}
$B/$ Résous les équations suivantes:
\setlength{\multicolsep}{0pt}
\begin{multicols}{3}
\begin{enumerate}
\item $2x+6=0$ 
\item $-7x+14=0$ 
\item $-3x-6=0$ 
\end{enumerate}
\end{multicols}

\end{document}