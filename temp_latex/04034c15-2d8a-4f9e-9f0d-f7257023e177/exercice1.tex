\indent Ton établissement organise chaque année des compétitions de football en prélude aux journées culturelles. Cette année, pour prendre compte l'équité genre, Le proviseur demande à ton Professeur de sports de constituer des équipes mixtes, c'est-à-dire des équipes composées de garçons et de filles. Toutes les équipes doivent comporter un même nombre de filles et un même nombre de garçons; le nombre de filles peut être différent du nombre de garçons. $80$ filles et $150$ garçons sont aptes pour jouer et tout(e) élève apte doit jouer dans une équipe.\\
Le professeur de sports voudrait former le plus grand nombre possible d'équipes mixtes. Il sollicite ton aide, toi élève en classe de quatrième($4^{eme}$), pour déterminer le nombre d'équipes mixtes qu'il pourra constituer, déterminer aussi la composition de chaque équipe en fille et en garçon et ensuite le nombre totale d'élèves par équipe.
\begin{center}
\begin{tabular}{|p{3cm}|p{3cm}|p{3cm}|p{3cm}|p{3cm}|}
\hline 
\textbf{Critères} & CM 1 & CM 2 & CM 3 & CP \\ 
\hline 
\textbf{Barème} & 2 pts & 2 pts & 2 pts & 2 pts\\ 
\hline 
\end{tabular} 
\end{center}