\indent Akpédzé et Makafui sont deux élèves jumelles qui vivent avec leur grand-mère à ANYRON. A l'occasion de la f\^ete de fin d'année, elles se préparent pour aller f\^eter chez leur maman à Kpalimé. Elles disposent d'une somme de 5000F pour leur voyage. Ne connaissant pas les frais de transport, elles se renseigne chez le chauffeur TOKPON qui leur donne les conditions de voyage: \\
\ding{118} les frais de transport d'un seul passager sont à 2000F\\
\ding{118} les deux jumelles ont le droit de tenir 25kg de bagages sans payer les frais de colis. Mais, si la masse de colis dépasse 20kg, elles vont payer comme frais de bagages 100F par kg le surplus.\\
Avec ces conditions, Akpédzé et Makafui se demande si leur somme de 5000F pourra les suffire ou pas à effectuer le voyage. 
Elles veulent voyager avec un petit sac de riz de 10kg, une valise de 12kg et un sac d'igname de 18kg. \\
A partir des calculs, dit en justifiant ta réponse si ces deux jumelles pourront ou pas effectuer leur voyage avec la somme dont elles dispose. \\
\begin{tabular}{|c|c|c|c|}
\hline 
Pertinence: (2,5pts) & Correction: (2,5pts) & Cohérence: (2pts) & Perfectionnement: (1pt)\\ 
\hline 
\end{tabular}