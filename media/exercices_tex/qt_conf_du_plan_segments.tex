Trace le segment [AB] tel que $AB = 9$ cm. Place un point E sur le segment [AB] tel que $BE = 7$ cm. Trace la droite $(D_1)$ perpendiculaire à (AB) et passant par le point E. Place ensuite un point C appartenant au segment [AB] tel que $AC = 5$ cm. Trace la droite $(D_2)$ perpendiculaire à (AB) et passant par le point C.

\begin{enumerate}
  \item Que peut-on dire des droites $(D_1)$ et $(D_2)$ ? Justifie ta réponse. \textbf{(0,5pt $\times$ 2)}
  \item Calcule la distance $BE$ et la distance $EC$. \textbf{(0,5pt $\times$ 2)}
  \item Justifie ensuite que le point E est le milieu du segment [BC]. \textbf{(0,75pt)}
  \item Vérifie que la droite $(D_1)$ est la médiatrice du segment [BC]. \textbf{(0,75pt)}
\end{enumerate}

\vspace{0.3cm}
\hfill \textbf{Figure : (2,5pt)}